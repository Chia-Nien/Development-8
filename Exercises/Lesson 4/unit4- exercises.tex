\documentclass[]{article}

\usepackage{enumitem}
\usepackage{geometry}[top = 1cm, bottom = 2cm, left = 1.5cm, right = 1.5cm]
\usepackage{listings}

%opening
\title{Development 8 - Exercises\\Unit 4}
\author{}
\date{}

\newcounter{ExerciseCount}
\setcounter{ExerciseCount}{1}

\newcommand{\functionEx}[3]{
  Implement a function\\\\
   \texttt{let #1 = #2}\\\\ #3
}

\newcommand{\exercise}[1]{\noindent \textbf{Exercise \theExerciseCount:}\\\\ #1 \addtocounter{ExerciseCount}{1}
}

\lstset{
  breaklines = true,
  basicstyle = \ttfamily,
  tabsize = 2
}

\begin{document}
\maketitle

\noindent


\exercise{
  \functionEx{filter}{<a>(predicate: (x: a) => boolean) => (l: List<a>): List<a>}{
    that inserts in the output list only the elements for which \texttt{predicate} returns true
}}\\

\exercise{
  \functionEx{map}{<a, b>(f: (x: a) => b) => (l: List<a>): List<b>}{
    that applies the function \texttt{f} to all the elements of \texttt{l} and returns a list containing the results.
}}\\

\exercise{
  \functionEx{fold}{<s, a>(f: (state: s) => (x: a) => s) => (init: s) => (l: List<a>): s}{
    that applies a function \texttt{f} to elements in the same position from \texttt{l}, threading an accumulator argument of type \texttt{s} through the computation.
}}\\

\exercise{
  \functionEx{apply}{<a, b>(f: (x: a) => b) => (x: a): b}{
    that applies function `f` to element `x`.
}}\\

\exercise{
  \functionEx{carry}{<a, b>(f: (x: a, y: b) => b) => (x: a): b}{
    that applies function `f` to element `x`.
}}\\

\exercise{
  \functionEx{mapFold}{<a, b>(f: (x: a) => b) => (l: List<a>): List<b>}{
    that implements \texttt{map} only using \texttt{fold}
}}\\

\exercise{
  \functionEx{filterFold}{<a>(predicate: (x: a) => boolean) => (l: List<a>): List<a>}{
    that implements \texttt{filter} only using \texttt{fold}
}}\\

\exercise{
  \functionEx{flatten}{<a>(l: List<List<a>>): List<a>}{
    that takes a list of lists and places all their elements it into a single one. Use \texttt{fold} to implement this function.
}}\\

\exercise{
  \functionEx{map2}{<a, b, c>(f: (x: a) => (y: b) => c) => (l1: List<a>) => (l2: List<b>): List<c>}{
    that applies the function \texttt{f} to the elements in the same position of two lists of equal length \texttt{l1} and \texttt{l2}.
}}\\

\exercise{
  \functionEx{fold2}{<s, a, b>(f: (state: s) => (x: a) => (y: b) => s) => (init: s) => (l1: List<a>) => (l2: List<b>): s}{
    that applies a function \texttt{f} to elements in the same position from \texttt{l1} and \texttt{l2}, threading an accumulator argument of type \texttt{s} through the computation.
}}\\

\exercise{
  \functionEx{zip}{<a, b>(l1: List<a>) => (l2: List<b>): List<Tuple<a, b>>}{
    that take two lists with the same length and creates a list of pairs containing the elements that are in the same position from both lists. Implement this function by using normal recursion and then by using \texttt{fold2}
}}\\

\exercise{
  \functionEx{map2Safe}{<a, b, c>(f: (x: a) => (y: b) => c) =>(l1: List<a>) =>\\ (l2: List<b>): List<Option<c>>}{
    that applies the function \texttt{f} to the elements in the same position of two lists\texttt{l1} and \texttt{l2}, possibly with different length. If an element of one list does not have a correspondent element in the second list, then the function returns \texttt{None}.\\\\
    
    \noindent
    \textbf{Example:}
    Summing the elements of \texttt{[1, 2, 3, 4]} and \texttt{[4, 5]} with \texttt{map2Safe} returns \texttt{[Some(5), Some(7), None, None]}
}}\\
\end{document}