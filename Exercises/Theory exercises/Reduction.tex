\documentclass[10pt,a4paper]{article}
\usepackage[utf8]{inputenc}
\usepackage[english]{babel}
\usepackage{amsmath}
\usepackage{amsfonts}
\usepackage{amssymb}
\usepackage{amsthm}
\usepackage{enumitem}
\usepackage{listings}
\usepackage{pdfpages}
\usepackage{comment}
\usepackage{mathpartir}
\usepackage[top=2cm, bottom=3cm, left=1.5cm, right=1.5cm]{geometry}

\newcommand{\tu}{\textunderscore}

\lstset{
	tabsize=2,
	basicstyle=\small\ttfamily,
	columns=fixed,
	showstringspaces=false,
	breaklines=true,
	showtabs=false,
	showspaces=false,
	showstringspaces=false,
}

\newcounter{ExerciseCount}
\setcounter{ExerciseCount}{1}

\newcommand{\functionEx}[3]{
  Implement a function
  
  \vspace{0.15cm}
  \texttt{let #1 = #2}
   
   \vspace{0.15cm}
   #3
}
\newcommand{\reductionEx}[3]{
  \noindent
  Given the following untyped lambda-calculus expression:
  
  \begin{equation*}
  #1
  \end{equation*}
  
  \noindent
  replace the requested terms with the elements from the expression in the following lambda-calculus rule that evaluates it:
  
  \begin{mathpar}
  #2
  \end{mathpar}
  
  \begin{equation*}
  #3
  \end{equation*}
}

\newcommand{\varRule}{
  \begin{mathpar}
  \mprset{flushleft}
  \inferrule*{ }
  {x \rightarrow_{\beta} x}
  \end{mathpar}
}

\newcommand{\lambdaRule}{
  \begin{mathpar}
  \mprset{flushleft}
  \inferrule*{ }
  {(\lambda x \rightarrow t) \; u \rightarrow_{\beta} t[x \mapsto u]}
  \end{mathpar}
}

\newcommand{\appRule}{
  \begin{mathpar}
  \mprset{flushleft}
  \inferrule*{ t \rightarrow_{\beta} t' \wedge u \rightarrow u' \wedge t' \; u' \rightarrow_{\beta} v }
  {t \; u \rightarrow_{\beta} v}
  \end{mathpar}
}

\newcommand{\emptyPlace}{\tu\tu\tu}
\newcommand{\exercise}[1]{\noindent \textbf{Exercise \theExerciseCount:}
  
  \vspace{0.15cm}
 #1 \addtocounter{ExerciseCount}{1}
}

\title{Development 8 - Theory exercises}
\date {  }


\begin{document}
\maketitle
\large
\textbf{Beta-reduction rules:}
\normalsize

\vspace{0.5cm}
\textbf{Variables:}
\varRule

\vspace{0.5cm}
\textbf{Function application:}
\lambdaRule

\vspace{0.5cm}
\textbf{Application:}
\appRule

\vspace{0.5cm}
\exercise{\reductionEx{
    (\lambda x \; y \rightarrow x \; y) \; (\lambda y \rightarrow x \; y)
  }{
    \mprset{flushleft}
    \inferrule*{ }
    {(\lambda x \rightarrow t) \; u \rightarrow_{\beta} t[x \mapsto u]}
  }{
    \left \lbrace
    \begin{array}{l}
      x = \dots \\
      t = \dots \\
      u = \dots \\
      t[x \mapsto u] = \dots
    \end{array} \right.
  }
}

\vspace{0.5cm}
\exercise{\reductionEx{
    (\lambda x \rightarrow \lambda x \rightarrow \lambda y \rightarrow y \; x) \; A
  }{
    \mprset{flushleft}
    \inferrule*{ }
    {(\lambda x \rightarrow t) \; u \rightarrow_{\beta} t[x \mapsto u]}
  }{
    \left \lbrace
    \begin{array}{l}
      x = \dots \\
      t = \dots \\
      u = \dots \\
      t[x \mapsto u] = \dots
    \end{array} \right.
  }
}

\vspace{0.5cm}
\exercise{\reductionEx{
    (\lambda f \; g \rightarrow f) (\lambda x \rightarrow y \; x)
  }{
    \mprset{flushleft}
    \inferrule*{ }
    {(\lambda x \rightarrow t) \; u \rightarrow_{\beta} t[x \mapsto u]}
  }{
    \left \lbrace
    \begin{array}{l}
      x = \dots \\
      t = \dots \\
      u = \dots \\
      t[x \mapsto u] = \dots
    \end{array} \right.
  }
}

\vspace{0.5cm}
\exercise{\reductionEx{
    (\lambda x \; y \rightarrow ((\lambda z \rightarrow z) \; x)) \; 0
  }{
    \mprset{flushleft}
    \inferrule*{ }
    {(\lambda x \rightarrow t) \; u \rightarrow_{\beta} t[x \mapsto u]}
  }{
    \left \lbrace
    \begin{array}{l}
      x = \dots \\
      t = \dots \\
      u = \dots \\
      t[x \mapsto u] = \dots
    \end{array} \right.
  }
}

\vspace{0.5cm}
\exercise{\reductionEx{
    (\lambda x \rightarrow (\lambda x \; y \rightarrow y \; x)) \; 0
  }{
    \mprset{flushleft}
    \inferrule*{ }
    {(\lambda x \rightarrow t) \; u \rightarrow_{\beta} t[x \mapsto u]}
  }{
    \left \lbrace
    \begin{array}{l}
      x = \dots \\
      t = \dots \\
      u = \dots \\
      t[x \mapsto u] = \dots
    \end{array} \right.
  }
}


\vspace{0.5cm}
\exercise{\reductionEx{
  ((\lambda x \; y \rightarrow y) \; 5) \; ((\lambda x \rightarrow x) \; 3)
}{
  \mprset{flushleft}
  \inferrule*{ t \rightarrow_{\beta} t' \wedge u \rightarrow u' \wedge t' \; u' \rightarrow_{\beta} v }
  {t \; u \rightarrow_{\beta} v}
}{
 \left \lbrace
 \begin{array}{l}
   t = \dots \\
   u = \dots \\
   t' = \dots \\
   u'= \dots \\
   v = \dots
 \end{array} \right. 
}}

\vspace{0.5cm}
\exercise{\reductionEx{
  ((\lambda f \; g \rightarrow f) \; (\lambda x \rightarrow x)) \; 5
}{
  \mprset{flushleft}
  \inferrule*{ t \rightarrow_{\beta} t' \wedge u \rightarrow u' \wedge t' \; u' \rightarrow_{\beta} v }
  {t \; u \rightarrow_{\beta} v}
}{
 \left \lbrace
 \begin{array}{l}
   t = \dots \\
   u = \dots \\
   t' = \dots \\
   u'= \dots \\
   v = \dots
 \end{array} \right. 
}}

\vspace{0.5cm}
\exercise{\reductionEx{
  (((\lambda f \; g \rightarrow g) \; ((\lambda x \; y \rightarrow \; y) \; 3)) \; (\lambda f \; g \rightarrow f))
}{
  \mprset{flushleft}
  \inferrule*{ t \rightarrow_{\beta} t' \wedge u \rightarrow u' \wedge t' \; u' \rightarrow_{\beta} v }
  {t \; u \rightarrow_{\beta} v}
}{
 \left \lbrace
 \begin{array}{l}
   t = \dots \\
   u = \dots \\
   t' = \dots \\
   u'= \dots \\
   v = \dots
 \end{array} \right. 
}}

\vspace{0.5cm}
\exercise{\reductionEx{
  (((\lambda x \; y \; z \rightarrow y \; z \; x) \; (\lambda x \rightarrow x)) (\lambda y \; x \rightarrow x \; y)) \; 3
}{
  \mprset{flushleft}
  \inferrule*{ t \rightarrow_{\beta} t' \wedge u \rightarrow u' \wedge t' \; u' \rightarrow_{\beta} v }
  {t \; u \rightarrow_{\beta} v}
}{
 \left \lbrace
 \begin{array}{l}
   t = \dots \\
   u = \dots \\
   t' = \dots \\
   u'= \dots \\
   v = \dots
 \end{array} \right. 
}}

\exercise{\reductionEx{
  (\lambda x \; y \rightarrow x) \; (((\lambda f \; g \rightarrow g) \; (\lambda x \rightarrow x)) \; 3)
}{
  \mprset{flushleft}
  \inferrule*{ t \rightarrow_{\beta} t' \wedge u \rightarrow u' \wedge t' \; u' \rightarrow_{\beta} v }
  {t \; u \rightarrow_{\beta} v}
}{
 \left \lbrace
 \begin{array}{l}
   t = \dots \\
   u = \dots \\
   t' = \dots \\
   u'= \dots \\
   v = \dots
 \end{array} \right. 
}}


\end{document}