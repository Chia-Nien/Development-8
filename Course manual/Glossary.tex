\newglossaryentry{fpvsimp}{
      name=fpvsimp
    , description={The learning objective stating that at the end of this course: the student \textbf{understands} concepts of functional vs imperative semantics.}
    , first={\textbf{understands} the fundamental semantic difference between functional and imperative programming. \texttt{(FP VS IMP)}}
    , symbol=\texttt{FP VS IMP}
    }


\newglossaryentry{red}{
	name=red
	, description={The learning objective stating that at the end of this course: the student \textbf{understands} reduction strategies such as $\rightarrow_\beta$.}
	, first={\textbf{understands} reduction strategies such as $\rightarrow_\beta$. \texttt{(RED)}}
	, symbol=\texttt{RED}
}


\newglossaryentry{typ}{
	name=typ
	, description={The learning objective stating that at the end of this course: the student \textbf{understands} the basics of a functional type system.}
	, first={\textbf{understands} the basics of a functional type system. \texttt{(TYP)}}
, symbol=\texttt{TYP}
}


\newglossaryentry{fpext}{
	name=fpext
	, description={The learning objective stating that at the end of this course: the student \textbf{can program} with the common extensions of a functional programming language wrt the basic lambda calculus, such as \texttt{let}, \texttt{if}, \textbf{let-rec}, unions, tuples, records, etc. The language of focus is Typescript.}
	, first={\textbf{can program} with the typical constructs of a modern functional language. The language of focus is Typescript. \texttt{(FP EXT)}}
, symbol=\textbf{FP EXT}
}
