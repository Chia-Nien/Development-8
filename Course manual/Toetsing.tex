\section{Assessment}
The course is tested with two exams: a written exam and a practical exam. The final grade is determined as follows: \\

\texttt{0.3 * writtenGrade + 0.7 * practicalGrade}

\textbf{Note:} both parts must be sufficient (i.e. \texttt{>= 5.5}) to pass the exam.

\paragraph*{Motivation for grade}
A professional software developer is required to be able to program code which is, at the very least, \textit{correct}.

In order to produce correct code, we expect students to show:
\begin{inparaenum}[\itshape i\upshape)]
\item a foundation of knowledge about how the semantics of the programming language actually work;
\item fluency when actually writing the code.
\end{inparaenum}

The quality of the programmer is ultimately determined by his actual code-writing skills, therefore the written exam will contain require you to write code. This ensures that each student is able to show that his work is his own and that he has adequate understanding of its mechanisms.



\subsection{Theoretical examination \modulecode}
The general shape of an exam for \texttt{\modulecode} is made up of a short series of highly structured open questions.
In each exam the content of the questions will change, but the structure of the questions will remain the same. Questions might include (but not limited to): apply the semantics of lambda calculus on a small function, determine the type of a functional program, determine the result of the execution of a functional program. A sample exam will be provided during the course.

\subsection{Practical examination \modulecode}
The practical exam requires to complete code snippets provided in the exam text. The code snippets contain simple functions covering the topics seen in class. A sample exam will be provided during the course.
